
% Preamble
\documentclass{assignment}

% Packages
\usepackage{minted}
\usepackage{dsfont}
\usepackage{colortbl}
\usepackage[linesnumbered,ruled,longend]{algorithm2e} \usepackage[colorlinks=true,linkcolor=blue]{hyperref}
\usepackage{forest}
\usepackage{multirow}

\SetKwInOut{Input}{Input} \SetKwInOut{Output}{Output} \SetKwProg{Fn}{Function}{\string:}{end} \SetKwFunction{algohw}{AlgoHw}
\SetKwFunction{algo}{FindRamanujam}

% Document
\begin{document}

    \maketitle

    \newpage

	% Q1
    \begin{homeworkProblem}
	\begin{enumerate}
	\item[1.a] Prove that there exist three constants $c_1, c_2, n_0$ such that for $n>n_0$, $c_2 n^3\leq n^3-3n^2-n+1 \leq c_1n^3 $. Obviously choose $c_1 = 2, n_0 = 1$,  $n^3-3n^2-n+1 \leq n^3 \leq c_1n^3 $. So $T(n) = \mathcal{O}(n^3)$.
	
	Choose $n_0 = 5$, then $3n^2 \leq \frac{3}{5}n^3$, $n \leq \frac{1}{25}n^3$. So $$n^3-3n^2-n+1 \geq n^3 - \frac{3}{5}n^3 - \frac{1}{25}n^3 + 1 \geq \frac{9}{25}n^3 + 1 \geq \frac{1}{5}n^3.$$
Choose $c_2 = \frac{1}{5}$, then $T(n) = \Omega(n^3)$. So  
	$n^3-3n^2-n+1 = \Theta(n^3)$
	\item[1.b] Prove that there exist two constants $c, n_0$ such that for all $n > n_0$, $n^2 < c 2^n$, which is same as $2ln(n) < cnln2$. Take the derivative, $
	\frac{2}{n} < cln2$, so when $n > n_0 = 2$, choose $c = 1$. As $n_0^2 = 2^{n_0}$ and $2^n$ will grow faster, $n^2 < c 2^n$ for specific $n_0$ and $c$ have been proved, so $n^2 = \mathcal{O}(2^n)$.
	
	\item[2.a] $f(n) = \mathcal{O}g(n)$. For $n > n_0 = 2$
	$$n\sqrt{n} \leq n*n - 1$$
	Choose $c = 1$, it satisfies $f(n) = \mathcal{O}g(n)$.
	
	\item[3.a] Such two functions do not exist. When $f(n) = \Theta(g(n))$, exist$n_0, c_1, c_2$ such that for $n > n_0$, $c_2g(n) \leq f(n) \leq c_1 g(n)$, so the condition of $f(n) = \mathcal{O}(g(n))$ must meet.
	\item[3.b] f(n) = $n^3$, g(n) = $n^2$.
	
	\item[4] $f_2(n) < f_3(n) < f_1(n) < f_4(n)$
	
	As shown in lecture $n > \sqrt{n} > logn$, 
	
	\begin{equation*}
	\lim_{n \to \infty} \frac{\sqrt{n}logn}{n \sqrt{logn}} =\lim_{n \to \infty} \frac{logn}{n}
	= 1 < \infty
	\end{equation*}
	so $f_2(n) = \mathcal{O}(f_3(n))$.
	
	As $(\sqrt{x} + \sqrt{y})^2 = x + y + 2\sqrt{xy} > (\sqrt{x + y})^2$, 
	$$\sum_{i=1}^n \sqrt{i} > \sqrt{\sum_{i=1}^n i}> \sqrt{\frac{n^2+n}{2}} > \sqrt{n^2} > \sqrt{n}logn$$
	So $f_2(n) = \mathcal{O}(f_1(n))$
	
	$$\sum_{i=1}^n \sqrt{i} < n\sqrt{n} < 24n^{3/2} + 4n + 6$$
	
	$f_1(n) = \mathcal{O}(f_4(n))$. 
	
	Then $f_2(n) < f_3(n) < f_1(n) < f_4(n)$ has been proved. 
	\end{enumerate}
    	
    \end{homeworkProblem}
    
    
    % Q2
    \begin{homeworkProblem}
    
    \begin{enumerate}
    \item[1.a] Begin with function $f(n)$ and assume $a = 2$
    

\begin{center}
\begin{forest} for tree={
    edge path={\noexpand\path[\forestoption{edge}] (\forestOve{\forestove{@parent}}{name}.parent anchor) -- +(0,-12pt)-| (\forestove{name}.child anchor)\forestoption{edge label};}
}
[f(n)
[f(n/b)
[f(n/$b^2$)
[f(n/$b^3$)[..][..] ]
[f(n/$b^3$) [..][..]  ]]
[f(n/$b^2$)
[f(n/$b^3$)[..][..]  ]
[f(n/$b^3$)[..][..]   ] ] ]
[f(n/b)
[f(n/$b^2$)
[f(n/$b^3$)[..][..] ]
[f(n/$b^3$) [..][..]  ]]
[f(n/$b^2$)
[f(n/$b^3$)[..][..]  ]
[f(n/$b^3$)[..][..]   ] ] ] ]
\end{forest} 
\end{center}

\item[1.b.1] As for each node, the input size is $n/b^{j}$. When $n/b^{j} = 1$, it is the depth of the tree. So depth: $log_bn$
\item[1.b.2] From the graph, the number of leaves at each depth $j$ is $a^j$.
\item[1.b.3] at each depth $j$, total cost: $a^jf(n / b^j)$
\item[1.b.4] \begin{equation*}
\begin{aligned}
T(n) &= \sum_{j = 0}^{log_bn} a^jf(n / b^j) 
= \sum_{j = 0}^{log_bn - 1} a^jf(n / b^j) + a^{log_bn}f(1) \\
&= \sum_{j = 0}^{log_bn - 1} a^jf(n / b^j) + n^{log_ba}f(1) 
= \sum_{j = 0}^{log_bn - 1} a^jf(n / b^j) +\Theta(n^{log_ba})
\end{aligned}
\end{equation*}
\item[2.a.1]

From $f(n)=\Theta (n^{log_ba})$, obtain that there exists three constants $n_0, c_1, c_2$ such that for all $n>n_0$ 

$$
  c_1 \cdot n^{log_ba}\leq f(n) \leq  c_2 \cdot n^{log_ba} 
$$
Then if replace n with $n/b^j$ and multiply with $a_j$ then calculate the sum,

$$
c_1 \cdot \sum_{j=0}^{\log_bn-1}a^j(\frac{n}{b^j})^{\log_ba} \leq g(n) \leq c_2 \cdot \sum_{j=0}^{\log_bn-1}a^j(\frac{n}{b^j})^{\log_ba}
$$
So
$$
  g(n)=\Theta\left(\sum_{j=0}^{\log_bn-1}a^j(\frac{n}{b^j})^{\log_ba}\right)
$$

\item[2.a.3] From the conclusion obtained in 2.a.2, 
    $$
g(n)=\Theta\left(\sum_{j=0}^{\log _{b} n-1} a^{j}\left(\frac{n}{b^{j}}\right)^{\log _{b} a}\right) = \Theta(n^{\log _{b} a} \log _{b} n)
$$

\item[2.b.2] 



\begin{align*}
 \sum_{j=0}^{\log _{b} n -1} a^{j}\left(n / b^{j}\right)^{\log _{b} a-\varepsilon}&=n^{\log _{b} a-\varepsilon} \sum_{j=0}^{\log _{b} n -1} a^{j} b^{-j \log _{b} a} b^{j \varepsilon}
  =n^{\log _{b} a-\varepsilon} \sum_{j=0}^{\log _{b} n -1} a^{j} a^{-j} b^{j \varepsilon} \\
  &=n^{\log _{b} a-\varepsilon}  \sum_{j=0}^{\log _{b} n -1} b^{\varepsilon j}
  =n^{\log _{b} a-\varepsilon} \frac{b^{\varepsilon\left(\log _{b} n\right)}-1}{1 - b^{\varepsilon}}\\
  &=n^{\log _{b} a-\varepsilon} \frac{n^{\varepsilon}-1}{b^{\varepsilon}-1}
\end{align*}
\item[2.b.3] From 2.b.1 and 2.b.2, 


\begin{align*}
  g(n) &= \mathcal{O}\left(\sum_{j=0}^{\log_bn-1}a^j(\frac{n}{b^j})^{\log_ba - \varepsilon}\right) = \mathcal{O}(n^{\log _{b} a-\varepsilon} \frac{n^{\varepsilon}-1}{b^{\varepsilon}-1})
\end{align*}

As \begin{align*}
n^{\log _{b} a-\varepsilon} \frac{n^{\varepsilon}-1}{b^{\varepsilon}-1} \leq n^{\log _{b} a-\varepsilon} \frac{n^{\varepsilon}}{b^{\varepsilon}-1}=n^{\log _{b} a} \frac{1}{b^{\varepsilon}-1},
\end{align*}
where $\frac{1}{b^{\varepsilon}-1}$ is a constant.
So,
$$
  g(n) = \mathcal{O}(n^{\log_ba})
$$

\item[2.c.1]
From $g(n) = \sum_{j = 0}^{log_bn - 1} a^jf(n / b^j) = \sum_{j = 1}^{log_bn - 1} a^jf(n / b^j) + f(n) > f(n)$, we could conclude that $g(n) = \Omega(f(n))$.

\item[2.c.2] 
From $af(n/b) \leq cf(n)$, we could set $x = n / b^{j-1}$,  so that $$a^j(n/b^j) = a^{j-1} \cdot a f(x/b) \leq a^{j-1} \cdot cf(x) = c a^{j-1}f( n / b^{j-1})$$
So that  we could repeat the steps above, until we get $f(n)$. 
$$
a^{j} f\left(n / b^{j}\right) \leq c \cdot a^{j-1} f(n/b^{j-1}) \leq c^2 \cdot a^{j-2} f(n/b^{j-2}) \leq \cdots \leq c^jf(n)
$$

\item[2.c.3]

$g(n) = \sum_{j = 0}^{log_bn - 1} a^jf(n / b^j) \leq  \sum_{j = 0}^{log_bn - 1} c^jf(n)\leq \frac{1}{1-c} f(n)$, so $g(n) = \mathcal{O}(f(n))$

\item[2.c.4]
So to prove that there exist three constants $c_1, c_2, n_0$ such that for $n>n_0$, $c_2 f(n)\leq g(n) \leq c_1 f(n) $. As proved in 2.c.3 with $c_1 = \frac{1}{1-c}$, only need to find $c_2 f(n)\leq g(n)$
\begin{align*}
  g(n)=\sum_{j=0}^{\log n-1} a^{j} f\left(n / b^{j}\right) 
  = a f\left(n\right)+\sum_{j=1}^{\log n-1} a^{j} f\left(n / b^{j}\right) 
  > f(n)
\end{align*}
so $g(n) = \Omega(f(n))$. Therefore, $g(n) = \Theta(f(n))$.
\item[3] Let $a\geq 1$ and $b > 1$ be constants, and n is an exact power of b. Then T(n) could be defined as 
\begin{align*}
	T(n) = \left\{
	\begin{aligned}
	&\Theta(1) \ \ \ & \rm{if } \quad n &= 1 \\
	& a T(n / b)+f(n)\ \ \  & \rm{if } \quad n &= b^i
	\end{aligned}
	\right.
\end{align*}
where i is a positive integer. Then from the conclusion we obtained,
\begin{align*}
	T(n) = \left\{
	\begin{aligned}
	&\Theta(n^{log_ba}) \ \ & f(n) &= \Theta(n^{log_ba})\\
	& \Theta(n^{log_ba})\ \ \  & f(n) &= O(n^{log_ba- \varepsilon}) \\
	& \Theta(f(n))\ \  & f(n) &\geq  \frac{a}{c}f(n/b)
	\end{aligned}
	\right.
\end{align*}

    \end{enumerate}
    \end{homeworkProblem}
    
    % Q3 
    \begin{homeworkProblem}
    
     \begin{algorithm}[H]
\Input{positive integer n}
\Output{all Ramanujam numbers smaller or equal to n} \BlankLine
\Fn{\algo{n}}{
i $\leftarrow$ 1 \;
cubeList $\leftarrow [ \ ] $\;
\While{i $<$ $\sqrt[3]{n}$}{ 
	push $i^3$ into cubeList\;
}
sumList  $\leftarrow [\  ] $\;
	\For{j $\leftarrow$ 1 \KwTo $\lceil \sqrt[3]{n} \  \rceil$}{
		\For{k $\leftarrow$ 1 \KwTo $\lceil \sqrt[3]{n}\  \rceil$} {
		push $j + k$ into sumList\;
	}	 		
}
\For {all elements in the sumList } {
	count the time they occur\;
	}
result  $\leftarrow $ elements that occur twice\;
\KwRet{result} }
\caption{Ramanujam numbers} 
\end{algorithm}
    \end{homeworkProblem}
    
    \begin{enumerate}
    \item[•] Find cube for all numbers that smaller than $ \sqrt[3]{n}$ = $\mathcal{O}(n^{1/3})$
    \item[•] Sum all pairs in the cubeList = $\mathcal{O}(n^{2/3})$
    \item[•] Find all numbers that occur twice = $\mathcal{O}(n)$
    \end{enumerate}
So the total time complexity should be $\mathcal{O}(n)$.
 
    
 
    %Q4
    \begin{homeworkProblem}
    The six pirates is P1, P2, P3, P4, P5, P6 and P1 is the senior-most pirate.
    
    We consider from the situation such that only one pirate survive. Because for those pirates  with proper thinking abilities, they would consider all the possible distribution methods to decide whether they should vote. 
    \begin{enumerate}
    \item Only P6 alive, P6 will have 300 coins.
    \item P5 and P6 alive. P5 will vote YES for any distribution he proposed (1 out of 2), so any distribution method could be accepted. Then distribution should be (P5, P6) have (300, 0) respectively, 
    \item P4, P5 and P6 alive. As P4 knows P6 will have no coin if P4 dies, P4 will give P6 one coin. Then 2 out of 3 will vote YES, P5 will not vote YES for P4 as he always could have better  income afterwards. Then distribution should be (P4, P5, P6) have (299, 0, 1) respectively.
    \item P3, P4, P5 and P6 alive. P3 need to get support from P5, as P5 obtain nothing in P4's distribution. Then with 2 out of 4 YES, P3's distribution could be accepted. Then distribution should be (P3, P4, P5, P6) have (299, 0, 1, 0) respectively.
    \item P2, P3, P4, P5 and P6 alive. P2 need to get support from P4 and P6, because they have nothing in P3's distribution. then 3 out of 5 YES will make it accepted. Then distribution should be (P2, P3, P4, P5, P6) have (298, 0, 1, 0, 1) respectively.
    \item All alive. P1 need to get support from P3 and P5. 3 out of 6 YES will make it accepted. 
    Then distribution should be (P1, P2, P3, P4, P5, P6) have (298, 0, 1, 0, 1, 0) respectively.
    \end{enumerate}
    
    
    \begin{table}[!ht]
    \centering
\begin{tabular}{|l|l|l|l|l|l|l|}
\hline
\begin{tabular}[c]{@{}l@{}}Number of \\ alive pirates\end{tabular} & \multirow{2}{*}{1} & \multirow{2}{*}{2} & \multirow{2}{*}{3} & \multirow{2}{*}{4} & \multirow{2}{*}{5} & \multirow{2}{*}{6} \\ \cline{1-1}
\begin{tabular}[c]{@{}l@{}}coins for \\ each pirates\end{tabular}  &                    &                    &                    &                    &                    &                    \\ \hline
1                                                                  & \textbackslash{}   & \textbackslash{}   & \textbackslash{}   & \textbackslash{}   & \textbackslash{}   & 298                \\ \hline
2                                                                  & \textbackslash{}   & \textbackslash{}   & \textbackslash{}   & \textbackslash{}   & 298                & 0                  \\ \hline
3                                                                  & \textbackslash{}   & \textbackslash{}   & \textbackslash{}   & 299                & 0                  & 1                  \\ \hline
4                                                                  & \textbackslash{}   & \textbackslash{}   & 299                & 0                  & 1                  & 0                  \\ \hline
5                                                                  & \textbackslash{}   & 300                & 0                  & 1                  & 0                  & 1                  \\ \hline
6                                                                  & 300                & 0                  & 1                  & 0                  & 1                  & 0                  \\ \hline
\end{tabular}
\end{table}
    
    \end{homeworkProblem}
    




\end{document}
