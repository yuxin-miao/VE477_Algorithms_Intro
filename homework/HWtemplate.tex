
% Preamble
\documentclass{assignment}

% Packages
\usepackage{minted}
\usepackage{dsfont}
\usepackage{colortbl}
\usepackage[linesnumbered,ruled,longend]{algorithm2e} \usepackage[colorlinks=true,linkcolor=blue]{hyperref}


\SetKwInOut{Input}{Input} \SetKwInOut{Output}{Output} \SetKwProg{Fn}{Function}{\string:}{end} \SetKwFunction{algohw}{AlgoHw}

% Document
\begin{document}

    \maketitle

    \newpage

	% Q1
    \begin{homeworkProblem}
	\begin{enumerate}
	\item
	Obviously $1 \leq k \leq n$, otherwise the statement would not be true. To prove, because the hash table has $n$ slots and the probability of $n$ keys to hash to any slot is equal, for each key, it has a probability of $\frac{1}{n}$ to hash to any slot. So the number of keys hash to a same slot follows a binomial distribution with parameter $n$ and $p$, where $p=\frac{1}{n}$. Then the probability for exactly k keys hash to a same plot is $$P_k = \left(\frac{1}{n}\right)^k \left(1-\frac{1}{n}\right)^{n-k} \binom n k.$$
	\item The probability of a slot to have $k$ keys is $P_k$. More than one slots may have k keys, but no slots would have more than k keys. Then, $P_k' = $the probability of at least one slot has $k$ keys and other slots have no more than $k$ keys, which is \textbf{smaller or equal} to the probability that at least one slot has $k$ keys. Because we have $n$ slots, the probability of only one slot has $k$ keys is $\binom n 1 P_k = nP_k,$ and this probability is \textbf{larger or equal} to the probability that at least one slot has $k$ keys. Through this two inequality, $P_k' \leq nP_k$.
	\item We have Stirling formula $n! \approx \sqrt{2\pi n}\left(\frac{n}{e}\right)^n,$ and $1 \leq k\leq n$
		\begin{equation*}
	\begin{aligned}
	P_k &= \left(\frac{1}{n}\right)^k \left(1-\frac{1}{n}\right)^{n-k} \frac{n!}{k!(n-k)!} \\
	&\approx \frac{\sqrt{2 \pi n}\left(\frac{n}{e}\right)^n}{\sqrt{2 \pi (n-k)}\left(\frac{n-k}{e}\right)^{n-k}k!} \left(\frac{1}{n}\right)^k \left(1-\frac{1}{n}\right)^{n-k} \\
	&=\sqrt{ \frac{n}{n-k}} \left(\frac{n}{n-k} \frac{n-1}{n}\right)^n \left(\frac{n-k}{e} \frac{1}{n}\frac{n}{n-1}\right)^{k} \frac{1}{\sqrt{2 \pi k}\left(\frac{k}{e}\right)^k} \\
	& < \sqrt{ \frac{n}{2 \pi k(n-k)}}   \frac{e^{-k}}{\left(\frac{k}{e}\right)^k} \\
	&< \frac{e^k}{k^k}
	\end{aligned}
	\end{equation*}
	
	
	
	\end{enumerate}
    	
    \end{homeworkProblem}
    
    
    % Q2
    \begin{homeworkProblem}

Suppose $G$ is an undirected graph $G$ with weighted edges and the weight of an edge $e$ is decreased where $e \notin T$, $e=(u,v)$.

 
     \begin{algorithm}[H]
\Input{this file}
\Output{nice algorithms in the homework} \BlankLine
\Fn{\algohw{this file}}{
download file\;
open file\;
compile file\;
\While{not at end of this document}{
read\; \uIf{understand}{
go to next line\;
current line becomes this one\; }
\uElseIf{want to know more on algorithms in \LaTeX} {refer to \href{http://tug.ctan.org/tex-archive/macros/latex/contrib/algorithm2e/doc/algorithm2e.pdf} {algorithm2e documentation}}
\Else {restart reading from the beginning\;} }
\For{exercise $\leftarrow$ 1 \KwTo 7}{ \If{algorithm is requested} {
solve the problem\;
A[$exercise$] $\leftarrow $ write the algorithm in \LaTeX\; }
}
\KwRet{A} }
\caption{Algorithms in the homework} \end{algorithm}
    
    \end{homeworkProblem}
    
    %Q3
    \begin{homeworkProblem}
    
    
    \end{homeworkProblem}
    




\end{document}
