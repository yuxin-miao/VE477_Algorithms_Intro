\documentclass[catalog.tex]{subfiles}

% do not write anything in the preamble

\begin{document}

\def\pbname{Generating Subsets} %change this, do not use any number, just the name

\section{\pbname} 

% only for overview, so short description (no more than 1-2 lines)
\begin{overview}
\item [Algorithm:] Backtracking~(algo.~\ref{alg:\currfilebase_a}), binaryCounting~(algo.~\ref{alg:\currfilebase_b}), grayCode~(algo.~\ref{alg:\currfilebase_c})
	% -	must match the label of the algorithm 
	% - when writing more than one algo use alg:\currfilebase_a, alg:\currfilebase_b, etc.
\item [Input:] A set of integers with length $n$
\item [Complexity:] $\mathcal{O}(2^n)$ with Backtracking, $\mathcal{O}(n\cdot 2^n)$ with binaryCounting and grayCode
\item [Data structure compatibility:] stack can be used with Backtracking, no specific data structure needed with binaryCounting 
\item [Common applications:] vertex cover, knapsack, set packing \cite{1}
\end{overview}


\begin{problem}{\pbname}
	Given a set of distinct integers, generate all possible subsets. 
\end{problem}


\subsection*{Description}
As specified in the overview, the input is a set of distinct integers. Although $set$ is defined as consisting of non-repetitive elements, distinct is emphasized to avoid misunderstanding. Input could also be a positive integer $n$, which represents the set $\{ 1, 2, ... ,n\}$. And this is could be included in the \textit{a set of distinct integers}. 

A more general input would be a set of \textit{any type items}, such as set $\{ a, b, c\}$. As the algorithm would be same as for a set of integers, this project will use a set of integers for simpler explanation. 

The output  is all the subsets of the given set $S$ with length $n$ (power set $P(S)$), so there are $2^n$ subsets in $P(S)$. For successful generating, the key is to establish a numerical sequence among all subsets and there are three primary alternatives \cite{1}
\begin{enumerate}
\item[•] \textit{Lexicographic order}: the most intuitive order for generating combinatorial objects, as human will generally use. For example, the eight subsets of $\{ 1, 2, 3\}$ in lexicographic order would be $\{ \}, \{1\}, \{1,2\}, \{1,2,3\}, \\ \{1,3\}, \{2\}, \{2,3\}, \{3\}$. As each time we find a previous number "back", the \textbf{backtracking} algorithm will find the subset in this order. 

\item[•] \textit{Binary counting}: Based on the binary number representation of the number of the subset. For example,for set $\{ 1, 2, 3\}$, the binary number representation from $0$ to $2^n-1 = 7$ would be $000, 001, 010, 011, 100, 101, 110,\\ 111$, which corresponds to $\{ \}, \{3\}, \{2\}, \{2,3\},  \{1\}, \{1,3\}, \{1,2\}, \{1,2,3\}$. As the subset could be decoded through the number from $0$ to $2^n-1 = 7$, the \textbf{binaryCoutning} algorithm would find subsets in this order. 

\item[•] \textit{Gray code}: A subset sequence as the adjacent subsets differ by insertion or deletion of exactly one item. The gray code for 3-bits is  $000, 001, 011, 010, 110, 111, 101, 100$. Compared with $binary counting$, such sequence could not be obtained directly. With this sequence, generate the subsets based on this sequence will have same procedure as based on $binaryCounting$. 
\end{enumerate}

% add comment in the pseudocode: \cmt{comment}
% define a function name: \SetKwFunction{shortname}{Name of the function}
% use the defined function: \shortname{$variables$}
% use the keyword ``function'': \Fn{function name}, e.g. \Fn{\shortname{$var$}}

\subsubsection{Backtracking}
The key idea is that with $n$ elements in the set, for each element, there are two choices: either include or exclude the element in the subset.
\begin{Algorithm}[\textbf{Backtracking}($set$)\label{alg:\currfilebase_a}]
\SetKwFunction{myFunc}{subsetRec}
	\Input{A set of distinct integers $set$}
	\Output{$result$, store all the subsets of $set$}
	\BlankLine
	\BlankLine
\Fn{\myFunc{$set$, $subset$, $result$, $index$}}{
	push $subset$ into $result$ \\
	\For {$i$ $\leftarrow$ $index$ \KwTo  $len(set)$}{
	push $set[i]$ into $subset$ \\
	\myFunc{$set$, $subset$, $result$, $index+1$} \\
	pop out the last element of $subset$ \cmt{exclude $set[i]$ and backtracking}\\
	}
	\Ret 
}
\myFunc{$set$, $[\ ]$, $[[\ ]]$, $0$} \\
\Ret $result$
\end{Algorithm}

To analyse time complexity, because not all the sub-problems would have roughly equal size, Master methodis not applicable. The number of execution operations could be represented as 
$$T(n) = \sum_{i=1}^{n-1} T(n-i) + c$$
which is caused by the $for$ loop that from $0$ to $len(set)$. As each term could be expanded following the same equation such that $$T(n-1) = T(n-2) + T(n-1)+...+T(1)$$
Then we could get 
\begin{align*}
T(n) &= T(n-1) + T(n-2) + ... + T(1) + c \\
&=\left( T(n-2) + T(n-3)+ ... + T(1) + c \right) + T(n-2) + ... + T(1) + c \\
&= 2\times T(n-1) \\
T(n-1) &= 2\times T(n-2) \\
&... \\
T(2) &= 2T(1) \\
T(1) &=T(0)
\end{align*}
Because the equation only differs from constant term, the equation above is acceptable. As we have $T(1) = \mathcal{O}(1)$, $$T(n) = 2^{n-1}T(1) = \mathcal{O}(2^n).$$

This result could also be obtained by a very intuitive induction. Such that each step there are two choices and  n elements (to take both choices) in total, the time complexity should be  $\mathcal{O}(2^n)$
\subsubsection{Binary Counting}
The number of all subsets of a set with size $n$ is $2^n$. Binary counter approach could be used, as any unique binary string of length $n$ represent a unique subset. So we could start from $0$ and end with $2^n - 1$, and for every bit that set as 1, push the number represent by this bit into subset. 
\begin{Algorithm}[\textbf{binaryCounting}($set$)\label{alg:\currfilebase_b}]
	\Input{A set of distinct integers $set$}
	\Output{$result$, store all the subsets of $set$}
	\BlankLine
	\BlankLine
$n \leftarrow$ $len(set)$ \\	
\For {$count$ $\leftarrow$ $0$ \KwTo $2^n - 1$}	{
$subset \leftarrow [\ ]$ \\
	\For {$bit$ $\leftarrow$ $0$ \KwTo $n$} 
	{
	\If{ $count \& (1 << bit)$\cmt{every time left shift 1 by $bit$}} {
	push $set[bit]$ into $subset$ \cmt{find the bit set as 1 in $count$} \\  
	}
	}
	push $subset$ into $result$ \\
}	

\Ret $result$
\end{Algorithm}

Time complexity for this algorithm is easier to analyse. The outer loop will execute for $2^n$ times and inner will execute for $n$ times. So the total time complexity is $\mathcal{O}(n\cdot 2^n)$

\subsubsection{Gray Code}
Recursion and iteration both could be used to generate gray code with given length $n$. As they have similar idea, here only the iteration method is presented. 

\begin{Algorithm}[\textbf{grayCode}($set$)\label{alg:\currfilebase_c}]
\SetKwFunction{myFunc}{IterationCode}
	\Input{A set of distinct integers $set$}
	\Output{$result$, store all the subsets of $set$}
	\BlankLine
	\BlankLine
\Fn{\myFunc{$n$}}{
$arr \leftarrow [[\ ]]$ \\ 
	push $"0", "1"$ into $arr[0], arr[1]$ \cmt{one-bit 0 or 1}\\
	\For {$i$ $\leftarrow$ $2$ \KwTo  $2^n$ \textbf{by} $\times 2$}{
	\For {$j$ $\leftarrow$ $i-1$ \KwTo $0$ \textbf{by} $-1$} {\cmt{the previous generated code is added again in reverse order} \\
	push $arr[j]$ into $arr$ \\
	}
	\For {$j$ $\leftarrow$ $0$ \KwTo $i$ \textbf{by} $+1$} {
	add $"0"$  ahead $arr[j]$ \cmt{0 to first half}\\
	}
	\For {$j$ $\leftarrow$ $i$ \KwTo $i*2$ \textbf{by} $+1$} {
	add $"1"$  ahead $arr[j]$ \cmt{1 to second half}\\
	}
	}
	\Ret $arr$
	}
$code \leftarrow$ \myFunc{$len(set)$}

\For {$j$ $\leftarrow$ $0$ \KwTo $2^n - 1$ }	{
$subset \leftarrow [\ ]$ \cmt{same procedure as in binaryCounting}\\
	\For {$bit$ $\leftarrow$ $0$ \KwTo $n$} 
	{
	\If{ $code[j] \& (1 << bit)$} {
	push $set[bit]$ into $subset$ \\  
	}
	}
	push $subset$ into $result$ \\
}	
\Ret $result$

\end{Algorithm}
Time complexity for this algorithm is $\mathcal{O}(n\cdot 2^n)$. 

For generating the code, the out loop will execute for $(n-1)$ times. For three inner loop, as they both use $i$ and $2 + 2^2 + ...+ 2^{n-1} = \mathcal{O}(2^n)$. So total time complexity for generating code is$\mathcal{O}(n\cdot 2^n) $. Generating the subsets based on the code is $\mathcal{O}(m\cdot 2^n)$. So we could get total time complexity $\mathcal{O}(n\cdot 2^n)$.

% include references where to find information on the given problem using latex bibliography
% insert references in the text (\cite{}) and write bibliography file in problem-nb.bib (replace nb with the problem number)
% prefer books, research articles, or internet sources that are likely to remain available over time
% as much as possible offer several options, including at least one which provide a detailed study of the problem
% if available include links to programs/code solving the problem
% wikipedia is NOT acceptable as a unique reference
\singlespacing
\newpage
\printbibliography[title={References.},resetnumbers=true,heading=subbibliography]

\end{document}
